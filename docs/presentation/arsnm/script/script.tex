\documentclass{article}

\usepackage[french]{babel}
\usepackage[a4paper, portrait, margin=20mm]{geometry}
\usepackage{caption}
\usepackage[T1]{fontenc}
\usepackage{amsmath, amsfonts, amssymb, mathtools}
\usepackage{stmaryrd}
\usepackage{fancyhdr}
\usepackage{graphicx}
\usepackage[ddmmyyyy]{datetime}
\usepackage{adjustbox}
\usepackage[explicit]{titlesec}
\usepackage{lastpage}

% page numeration
\pagestyle{fancy}
\fancyhf{}
\renewcommand{\headrulewidth}{0pt}
\fancyfoot[R]{\thepage/\pageref{LastPage}}

% document info
\makeatletter
\title{Script de Pr\'esentation}
\date{\today}
\newcommand{\matiere}{TIPE}
\newcommand{\classe}{MP\textsuperscript{*} }
\author{Arsène MALLET}

% header
\fancypagestyle{firstpage}{
    \fancyhead[L]{\@author}
    \fancyhead[C]{\classe - \matiere}
    \fancyhead[R]{\@date}
}

% slide counter / section reformater
\titleformat{\section}
{\Large\bfseries}
{Slide \thesection}{1em}{#1}

% timer command
\usepackage{xparse}

\ExplSyntaxOn
\NewDocumentCommand{\duration}{m}
 {
  \didou_duration:n { #1 }
 }

\fp_new:N \l_didou_duration_hrs_fp
\fp_new:N \l_didou_duration_min_fp
\fp_new:N \l_didou_duration_sec_fp

\cs_new_protected:Nn \didou_duration:n
 {
  \fp_compare:nTF { #1 < 60 }
   {% easy case
    \fp_eval:n { round(#1,3) }
   }
   {
    \fp_compare:nTF { #1 < 3600 }
     {% only minutes
      \didou_duration_minutes:n { #1 }
     }
     {% hours
      \didou_duration_hours:n { #1 }
     }
   }
 }

\cs_new_protected:Nn \didou_duration_minutes:n
 {
  \fp_set:Nn \l_didou_duration_min_fp { trunc( #1/60,0 ) }
  \fp_set:Nn \l_didou_duration_sec_fp
   {
    round( #1-\l_didou_duration_min_fp * 60,0 )
   }
  % now print
  \fp_eval:n { \l_didou_duration_min_fp }
  :
  \fp_compare:nT { \l_didou_duration_sec_fp < 10 } { 0 } % two digits
  \fp_eval:n { \l_didou_duration_sec_fp }
 }
\cs_generate_variant:Nn \didou_duration_minutes:n { V }

\cs_new_protected:Nn \didou_duration_hours:n
 {
  \fp_set:Nn \l_didou_duration_hrs_fp { trunc( #1/3600,0 ) }
  \fp_set:Nn \l_didou_duration_min_fp { #1 - \l_didou_duration_hrs_fp * 3600 }
  % print the hours
  \fp_eval:n { \l_didou_duration_hrs_fp } :
  % go to minutes
  \fp_compare:nT { \l_didou_duration_min_fp < 600 } { 0 }
  \didou_duration_minutes:V \l_didou_duration_min_fp
 }

\ExplSyntaxOff

\begin{document}

\thispagestyle{firstpage}

\begin{center}
    \huge\bfseries{\@title}
\end{center}

\section{Page initiale}

Bonjour [mesdames et messieurs] du jury, je suis ici aujourd'hui pour vous
présenter ma soutenance de TIPE. Cette année, moi et deux camarades avons travaillé
sur la compression de données, et plus particulièrement son application aux images. 

\section{Introduction}

Tout d'abord, commençons par \textbf{définir} ce qu'est la compression. La compression
est un \textbf{procédé transformant une suite de bits $A$ en une suite de bits $B$ plus 
courte pouvant restituer les mêmes informations, ou des informations voisines.}

La notion d'informations voisines intervient lorsque l'on évoque des \textbf{deux types de compressions différent} :
la compression \textbf{avec perte} (que l'on appelle \textit{lossy} en anglais) et la compression \textbf{sans perte}
(\textit{lossless}). Nous verrons au cours de cette présentation comment choisir entre ces deux type et comment on peut
également les faire fonctionner ensemble.

Dans le thème de TIPE de cette \textbf{la ville}, la compression est une notion très présente, puisque
\textbf{la nécessité} constante d'échange d'informations fait qu'il y a beaucoup de données à traiter et
\textbf{la compression} permet de \textbf{de stocker et traiter efficacement toutes ces données}.

\section{Procédés de compression}

Afin de compresser des données, il existe \textbf{tout un tas de différents procédés}. Dans la compression
\textit{lossless} on retrouve encore deux types, que sont les compressions \textbf{entropiques} et \textbf{algorithmiques}.
La compression entropiques supposent des \textit{a priori} sur ce que l'on compresse, comme des \textbf{tableaux de fréquence}
ou des \textbf{tables de codage} comme on le verra plus tard, tandis que les compressions algorithmiques ne nécessitent que de connaître
la méthode de compression utilisés afin de décompressé.

De manière générale, les compressions fonctionnent par réorganisation des données, et par l'application de transformées mathématiques
permettant de stocker l'information de manière plus optimale.

C'est le cas de l'algorithme JPEG, le format d'image le plus utilisés et celui sur lequel nous avons basé nos recherches, 
qui utilisent beaucoup des procédés les plus populaires pour compressé des images. Nous verrons son implementation 
au fur et à mesure de la présentation.

\section{Entropie et Codage Optimal (1)}

Lorsqu'il est question de codage en compression, la théorie mathématique qui définit les fondements de ces codages est la théorie
de l'information notamment fondé par Claude Shannon en 1948. Cette théorie probabiliste, qui quantifie l'information d'un ensemble
de messages, se base sur la notion d'entropie, définit ainsi : [lire définition du diapo]

\section{Entropie et Codage Optimal (2)}

Voici quelques définitions importantes de la théorie de l'information, dont nous aurons besoin pour la suite : 
[lire et expliquer les définitions] \\
Rq: l'extension $C^+$ est l'application qui à $\Omega^*$ associe la concatenation des codes.

\section{Entropie et Codage Optimal (3)}

Une famille de code uniquement décodable tel que définit précédemment est celle des code préfixe : [lire définition]

[Présenter l'exemple]

\section{Entropie et Codage Optimal (4)}

\end{document}