\documentclass{beamer}

\usepackage[french]{babel}
\usepackage{caption}
\usepackage[T1]{fontenc}
\usepackage{amsmath, amsfonts, amssymb}
\usepackage{stmaryrd}
\usepackage{fancyhdr}
\usepackage{lastpage}
\usepackage{lipsum}
\usepackage{graphicx}
\usepackage[ddmmyyyy]{datetime}
\usepackage{adjustbox}
\usepackage[explicit]{titlesec}
\usepackage{color, soul}
\setulcolor{red}

\setbeamertemplate{footline}{\insertframenumber}


\title{La Compression de Donn\'ees}
\subtitle{Appliquée aux Images}
\author{Ars\`ene MALLET}
\date{}
\institute{Numéro Candidat : 14873}

\begin{document}

\frame{\titlepage}

\begin{frame}
    \frametitle{Introduction}
    \small
    \begin{itemize}
        \item \textbf{Th\'eorie de l'information} $\rightarrow$ Transmission d'\textbf{information maximale} avec une \textbf{taille minimal}
        \item Deux types de compression : \textbf{avec} perte (lossy), et \textbf{sans} perte (lossless).
        \item 
    \end{itemize}
\end{frame}
\end{document}